
\documentclass{book}
\usepackage{cite}
\usepackage{chapterbib}
\usepackage{amsmath}
\usepackage{url}
\usepackage{pdfpages}
\setcounter{secnumdepth}{-1}
\begin{document}
	\title{The effect of Changes of Rainfall and Average Wind Speed effects on Climate}
	\author{Dr Barry D.O. Adams \\ barry.david.adams@gmail.com}
	\maketitle
	\section{Abstract}
		In this work, we show that a reduction in climate sensitivity due extra evaporation is a better match to climate sensitivity than without evaporation. We also compare the measured increase in rainfall, with our estimate of climate sensitivity, rainfall increase, with the IPCC's $2.45 W/m^2$ climate forcing. The rainfall is a good match, but expected temperature increase in the years since $1850$, becomes only $0.68 $ Kelvin. We also estimate the increase in evaporation of sea water, due to changes in global average wind speed) and show that the decrease in average wind speed from $1960$ to $2010$, we estimate this leads to decreased evaporation and thus an increase of temperature of around $0.9$ Kelvin.
		
	\section{Keywords} Average Wind, Evaporation, Climate Sensitivity, Rainfall
	
	\section{Introduction}
		In our last paper, Estimation of atmospheric IR absorption \cite{Adams1}, we saw that evaporation absorbs heat from the earth surface, and radiates it in the upper atmosphere,
		effective resulting in a cooling that increases with any surface temperature increase, this has the effect of reducing but not reversing any heating. Since rainfall is very measurable, we can compare the three numerical quantities of temperature and rainfall, then compare with the climate forcing ($CO_2$ greenhouse absorption in Watts per square meter) quoted in the literature, then see if the climate sensitivity is a better match, with or without the extra rainfall.  
		
		But there are two input factors to the
		amount of evaporation from any body of water, the temperature which we looked at, and the wind speed. To what extend does wind speed effect evaporation and thus
		temperature. We will also tentatively quantify this effect we will look at in this paper. 
		
	\section{Climate Sensitivity and Rainfall, reprise}
	
	The usual model of climate sensitivity is to use the Stefan Boltzmann equation to assume the increase in heat absorption all goes into the the heat balance between incoming and outgoing radiation \cite{wikisense}
	
	\begin{equation}
		\Delta F_{2xCO_2} = \frac{dF}{dT} \Delta T_{2xCO_2}  = 4\sigma T^3 \Delta T_{2xCO_2} 
	\end{equation}
	However we see an additional heat loss mechanism evaporation. At present an average rainfall of $39$ inches lands upon the earth surface annually \cite{rainfall}. But rainfall increases $7\%$ for each degree warmer it gets \cite{rainrise}.
	Using the mean temperature of $17.5$, and the usual heat of vaporization of water, we multiply the volume of water falling per square meter, $970kg$, by the specific heat of water times the average $82.5$ degree heating plus the heat of vaporization, gets $2.526GJ$ per year, or $80.11$ Watts per square meter. The evaporated water mostly condense at height, so at least half is radiated upwards, (more depending on the height of the cloud, as the horizon will be a little below horizontal. We call $E_{wc}$ the energy emitted by the water cycle . Adding radiated water cycle to the energy balance derivative equation gives. Using the maxim that a $1$ Kelvin increase of temperature leads a $7\#$ increase in rainfall \cite{rainrise}. 	
		\begin{equation}
		\Delta T_{2xCO_2} = \frac{ \Delta F}{ 4\sigma T^3 + 0.07 * E_{evap} exp( T-17.5)*0.07)}
	\end{equation}
	For our $1.1$ Watts predicted in our last paper \cite{Adams1} of additional $CO_2$ absorption from doubling $CO_2$, we get a warming of just $.39 C$. This $2.8$ times lower than the \cite{wikisense} Climate sensitivity in Wikipedia. In our previous paper \cite{Adams1} which expected all the condensation energy to dissipate to space, which could be true if the heat emission is primary transferred by convection to high attitudes. But if the heat is radiated from clouds, by radiation, it transfers roughly $50\%$ to space and $50\%$ to the ground. In a later section we will compare results on trends in rainfall over the last hundred years with both data on temperature. The best match is for the $1/2.8$ climate sensitivity.
	
	
	\section{Global Rainfall vs Temperature}
	
	Both global average rainfall and temperature have increased in during the 20th Century. According to NASA and the EPA \cite{EPARain} global average rainfall has increased at a rate of $0.1$ inches per decade.
	Meanwhile temperature has increase at a rate of $0.08 C$ per decade \cite{GlobTemp}.
	
   120 years of $0.1$ inches per decade is $2.43 cm$ extra per annum and multiplying by water's empalthy of Vaporisation $2,250,000 J/Kg$, then dividing by second in a year, and ten to get litres. We get $1.81 W/m^2$ as the average thermal energy needed to evaporate that extra much water.
   
   If rainfall energy goes $2.8$ parts to 1 into temperature as equation 2, then the global warming is $0.65$  Kelvin or Celsius since 1900, and input forcing heat is $1.81*(1+1/2.8)= 2.45 W/m^2$. This is very close to the IPCC figure of $2.43 W/m^2$ in \cite{Radiative}. It should be noted that, climate sensitivity without our rainfall cooling is about $1.0$, \cite{wikisense}, and that would predict a heating of $2.5C$, $150\% $ much greater than is seen in observations. However for our figures to be accurate we must account for $0.31$ Kelvin of temperature difference. We will do so (approximately) in our next section looking at the effect of average wind speed on rainfall.
   
   \section{Average Wind Speed and rainfall}
   
   	Evaporation of water $g_s$ in kilograms per second, depends, on air speed $v$, surface area $A$, the current humidity $x$ and the maximum humidity $x_s$ of saturated air of the same temperature. A simple empirical formula is given in \cite{EngTB1}, is shown below.
   \begin{equation}
   	g_s = (25 + 19v) A (x_s - x)/3600 
   \end{equation}
   Since the velocity appears linearly in the formula, average over different velocities at different locations or times, will effect any formula on evaporation amounts linearly. Now the fraction of the humidity of air, will vary, in  difficult to calculate way, with location and time. We thus turn to observations to see the effect of wind speed on rainfall. In \cite{Bretherton} Brethetrton and Back, observe wind speeds and rain fall in the Pacific ocean, and find a good correlation. We take their data points from their 8 graphs at latitude 10N, and longitudes 160E, 175E, 170W, 155W, 140W, 125W, 110W and 95W. Combining the data points and preforming
   linear regression, we get.
   
   \begin{equation}
    p = 3.124 * w - 1.582 
   	\end{equation}
   Where $p$ is the precipitation in $mm/day$ and $w$ is the wind speed in $m/s$. The $R^2$ of the regression is $0.709$, the $P$ value, the probability that the rain fall is uncorrelated with wind speed, is a tiny
   $4.2*10^{-13}$ so they are very correlated.
   
   Since the wind speed enters equation 3, linearly, we can assume the global average rain fall also depends approximately linearly in global wind speed. It is interesting that the intercept on the wind axis, has negative rain for zero wind, while in equation 3, evaporation is positive with no wind. We can ignore the intercept by looking the difference in average wind speed for different years. 
   
   Assuming rainfall continues to fall off as $7\%$ per degree of temperature reduction \cite{rainrise}, a geometric factor for the global rainfall versus the amount above at $10N$ is
   
   \begin{equation}
   	G = \frac{ \int_{-\pi/2}^{\pi/2}d\theta \int_0^{2\pi} d\phi exp(0.07 *(T(\theta)-T( 180 * 10/\pi))\cos \theta }{ \int_{-\pi/2}^{\pi/2}\theta \int_0^{2\pi} d\phi cos \theta d\theta }
   \end{equation}
Thus,
\begin{equation}
	G = \frac{1}{2}\int_{-\pi/2}^{\pi/2}d\theta  exp(-0.07 T(\theta))cos \theta
\end{equation}
   
      The average temperature of the surface of the earth at latitude $\theta$, is recorded by Fuelner at al \cite{Feulner}, in reproduced in the table below. Computing $G$ by numerical integration we find $G = 0.59262$
      
      \begin{table}
      	\caption{Reconstructed 1961-1990 Average Surface Air Temperature By Latitude}
      	\centering
      	\begin{tabular} { c c}
      		\hline\hline
      		Latitude & Temperature Celsius \\ 
      		\hline
      		-90 & -20 \\
      		-80 & -15 \\
      		-70 & -8 \\
      		-60 & -1 \\
      		-50 & 10 \\
      		-40 & 16 \\
      		-30 & 20 \\
      		-20 & 24 \\
      		-10 & 26 \\
      		0 & 27 \\
      		10 & 27 \\
      		20 & 25 \\
      		30 & 20 \\
      		40 & 17 \\
      		50 & 7 \\
      		60 & -1 \\
      		70 & -8 \\
      		80 & -15 \\
      		90 & -17 \\
      		\hline
      	\end{tabular}
      \end{table}    
      
      
   Using the number of seconds in a day, and the empalthy of vaporisation of water, we arrive at.
   
   \begin{equation}
   	\Delta F = 3.124 \Delta w * 2.25*10^{6} / (2*60*60*24)  = 4.06 w
   \end{equation}
   Where $F$ is half the energy of evaporation of water of each square meter. This shows the forcing effect of wind speed on climate is quite large.
   
   We didn't find a global data set of average wind speeds over time, but find two usefully factoids, \cite{Horizion}  Dogshaun tells us that from 1960 to 2010, global wind speed has declined by $0.3$ miles per hour each decade.
   While \cite{SciAmWind} Scientific American, tells us from 2010, global average wind speed has increased from $7 mph$ to $7.4 mph$.
   
   Let first us consider the heat effect of the fifty years of stilling from 1960. $\Delta F $ for the period is $2.68 W/m^2$ and the effect is an increase in heat. With the climate sensitivity of $1/2.8$ we predicted at the beginning of this paper, (half that in our last paper \cite{Adams}). The forcing of $2.68 W/m^2$ is equivalent to heating of $0.95K$ including the $G$ factor, this would explain almost all have the recent warming.
   
   For the years $2010$ to $2020$, we get $Delta F $ is $.725$ so there is a temperature decrease of $0.15K$ enough to cancel the average $CO_2$ heating for the decade quoted in \cite{GlobTemp}. This matches observations in  \cite{GlobTemp}, that global temperature peaked in $2016$. Of course there will be other effects on temperature including volcanos and El Nino events in each particular year.
   
           	
   \section{Conclusions}
   
   In this paper we return to our estimate of the effect of rainfall on climate sensitivity in iour last paper \cite{Adams1}, we correct our assumption that all the heat released by condensation goes is radiated to space, with the
   assumption that half is radiated to space and half to the surface. We then find that $1$ part to $2.8$ of radiative forcing goes into the temperature, while the $2.8$ parts going into increased rainfall. We look at the the increase average rainfall since 1900 and find this matches what is expected from the IPCC radiative forcing, \cite{Radiative} of $2.54 W/m^2$, we then get a temperate increase from $CO_2$ of $0.65K$ this is around $2/3$ of the measured amount. We then look at the other input variable to evaporation of water, wind speed. We find the that double the missing $0.95K$ can be explained by the slowing of average wind speed, provided the average forcing for the Pacific $10N$ can be extrapolated over the world. The regression fit for rain versus wind speed so performed over limited data and at one latitude, so its accuracy could be improved with further research, and using more global data.
   
    It should be noted that the $2.54 W/m^2$ forcing quoted by the IPCC \cite{Radiative} would lead to around $2.5$ Kelvins of warming since 1900, if the rainfall didn't effect the climate sensitivity, and the Wikipedia \cite{wikisense} figure remained. Thus our hypothesis greatly improves on the standard warming figures, moving it down to $0.68$ Kelvin. Our estimate of rainfall due to the global change in average wind speed, is comparable at $0.95$ Kelvin,
     thus the effects of wind-speed seem slightly greater than a the effects of $CO_2$ radiative forcing.
    
     Since figures are available for both wind and rain, more detailed calculations might in future show better temperature and wind speed factors in climate. Average wind speeds are highly variable over time. It should be noted that $0.34 W/m^2$ of forcing in \cite{Radiative} is due to halo-carbons which are no longer produced so $0.12K$ of heating will gradually disappear from the earth.
     
           		
   Our note here is short, and uses a narrow dataset of rain versus wind-speed. Metrology data is measurable for wind, rain and temperature, we recommend that studies are produced using as much data as possible, to
   confirm or disfavour, both the expects of temperature and wind-speed on rainfall, and show or not, how this rainfall amount then effects global (and at each latitude) temperatures.
   
   Because rainfall and evaporation happen the most in hotter regions of the world, our effects, would lead to the hotter counties being less effected by climate change, while zones of moderate temperature would grow hotter by a greater amount.
           		
	\begin{thebibliography}{99}
		\bibitem{EPARain} Climate Change Indicators: US and Global Precipitation\url{https://www.epa.gov/climate-indicators/climate-change-indicators-us-and-global-precipitation}
	    \bibitem{EngTB1} Engineering Toolbox; Evaporation from Water Surface \url{https://www.engineeringtoolbox.com/evaporation-water-surface-d_690.html}
	    \bibitem{Adams1} B Adams, Estimation of Atmospheric IR Absorption. Current Advances in Geography, Environment and Earth Sciences Vol. 3, 15 April 2022 , Page 106-112  DOI: 10.9734/bpi/cagees/v3/15971d
	    \bibitem{ClimateGov} Climate Change: Global Temperature \url{https://www.climate.gov/news-features/understanding-climate/climate-change-global-temperature}
	    \bibitem{GlobTemp} National Center for Environmental Information,  Annual 2020 Global Climate Report \url{https://www.ncei.noaa.gov/access/monitoring/monthly-report/global/202013}
	    \bibitem{Radiative} V. Ramaswamy, Radiative Forcing of Climate Change \url{https://www.ipcc.ch/site/assets/uploads/2018/03/TAR-06.pdf}
	    \bibitem{rainrise} \url{https://phys.org/news/2018-05-higher-temperature-heavier.html}
	    
	    \bibitem{Bretherton} L. E. Back, C. S. Bretherton. The relationship between wind speed and precipitation in the Pacific ITCZ, Journal of Climate 18(20) DOI 10.1175/JCLI3519.1
	     \url{https://www.researchgate.net/publication/249611650_The_Relationship_between_Wind_Speed_and_Precipitation_in_the_Pacific_ITCZ}
	    \bibitem{Horizion} J, Dodgshun, The stilling: global wind speeds slowing since 1960, \url{https://ec.europa.eu/research-and-innovation/en/horizon-magazine/stilling-global-wind-speeds-slowing-1960}
	    \bibitem{SciAmWind} The World’s Winds Are Speeding Up \url{https://www.scientificamerican.com/article/the-worlds-winds-are-speeding-up/}
		\bibitem{Mourya} Prof A. Mourya, Engineering Physics, Lecture Notes, \url{https://sites.google.com/site/puenggphysics/home/unit-i/relation-between-einstein-s-a-and-b-coefficient} (2015)
		\bibitem{Tokmakoff} Tokmakoff, Absolute Cross Sections, Lecture Notes, \url{http://home.uchicago.edu/~tokmakoff/TDQMS/Notes/4.3._Abs_Cross-Sec_2-12-08.pdf}
		\bibitem{Brit1} Britannica, T. Editors of Encyclopaedia (Invalid Date). Planck's radiation law. Encyclopedia Britannica. \url{https://www.britannica.com/science/Plancks-radiation-law}
		\bibitem{KW} Kenneth Wood,University of St Andrews , Nebulae, Lecture 8, Line Widths \url{http://www-star.st-and.ac.uk/~kw25/teaching/nebulae/lecture08_linewidths.pdf}
		\bibitem{HITRANUnits} HITRAN Definitions and Units \url{https://hitran.org/docs/definitions-and-units/}
		\bibitem{Goldman} Einstein A-coefficients and statistical weights for molecular
		absorption transitions in the HITRAN database, Marie Simeckova, David Jacquemarta, Laurence S. Rothmana,
		Robert R. Gamacheb, Aaron Goldmanc. \url{https://lweb.cfa.harvard.edu/atmosphere/publications/2006-EinsteinA-JQSRT-98.pdf}
		\bibitem{Feulner} G. Feulner, S. Rahmstorf, A. Levermann, S. Volkwardt, On the Origin of Surface Air Temperature Difference between Hemispheres in the Earth's Present-Day Climate,
		pg 7136-7130 Vol 26 Journal of Climate (2013)
		\bibitem{Lente} Lente, G., Ősz, K. Barometric formulas: various derivations and comparisons to environmentally relevant observations. ChemTexts 6, 13 (2020). https://doi.org/10.1007/s40828-020-0111-6
		\bibitem{wikiaugust} \url{https://en.wikipedia.org/wiki/Vapour_pressure_of_water}
		\bibitem{calcsource} \url{https://github.com/badams77-cpu/agw}
		\bibitem{green} Recent advances in measurement of the water vapour continuum in the far-infrared spectral region
		Paul D. Green
		, Stuart M. Newman
		, Ralph J. Beeby
		, Jonathan E. Murray
		, Juliet C. Pickering
		and John E. Harries
		June 2012 https://doi.org/10.1098/rsta.2011.0263
		\bibitem{wikisense}	\url{https://en.wikipedia.org/wiki/Climate_sensitivity}
		\bibitem{rainfall} \url{https://en.wikipedia.org/wiki/Earth_rainfall_climatology}
		\bibitem{rainrise} \url{https://phys.org/news/2018-05-higher-temperature-heavier.html}
		\bibitem{ACSCO2} \url{https://www.acs.org/content/acs/en/climatescience/atmosphericwarming/radiativeforcing.html}
		\bibitem{ACSsense} \url{https://www.acs.org/content/acs/en/climatescience/atmosphericwarming/climatsensitivity.html}
		\bibitem{Ency} Richard Tuckett, in Encyclopedia of Analytical Science (Third Edition), 2019 /url{https://www.sciencedirect.com/topics/chemistry/greenhouse-gas}
		\bibitem{ACSWater}  "It's Water Vapor, Not the CO2". American Chemical Society. \url{https://www.acs.org/content/acs/en/climatescience/climatesciencenarratives/its-water-vapor-not-the-co2.html}
		\bibitem{NASAForcing}  "Water vapour confirmed as Major Player in climate change" \url{https://www.nasa.gov/topics/earth/features/vapor_warming.html}
		\bibitem{Varan} Varanasi P. Infrared absorption by water vapour in the atmospheric window. In Modelling of the Atmosphere 1998 Aug 24 (Vol. 998, pp 213-230). Internation Society for Optics and Photonics.
		\url{https://doi.org/10.1117/12.975629}
		\bibitem{Rosenfeld} Rosenfeld J.E. A simple parametrization of ozone infrared absorption for atmospheric heating rate calculations. Journal of Geophysical Research. Atmospheres. 1991 May 20; 96(D5):9065-74.
		\bibitem{Ridgway}  Ming-Dah Chou, David P. Kratz and William Ridgway 
		Journal of Climate
		Vol. 4, No. 4 (April 1991), pp. 424-437 (14 pages)
		Published By: American Meteorological Society
		Journal of Climate    \url{https://www.jstor.org/stable/26196300}
		\bibitem{IPCC} United Nations, Climate Reports \url{https://www.un.org/en/climatechange/reports}
	\end{thebibliography}
	
\end{document}
			